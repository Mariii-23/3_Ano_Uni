% Created 2022-05-04 Wed 10:01
% Intended LaTeX compiler: pdflatex
\documentclass[11pt]{article}
\usepackage[utf8]{inputenc}
\usepackage[T1]{fontenc}
\usepackage{graphicx}
\usepackage{longtable}
\usepackage{wrapfig}
\usepackage{rotating}
\usepackage[normalem]{ulem}
\usepackage{amsmath}
\usepackage{amssymb}
\usepackage{capt-of}
\usepackage{hyperref}
\author{Mari§}
\date{\today}
\title{Teste Modelo}
\hypersetup{
 pdfauthor={Mari§},
 pdftitle={Teste Modelo},
 pdfkeywords={},
 pdfsubject={},
 pdfcreator={Emacs 27.2 (Org mode 9.6)}, 
 pdflang={English}}
\begin{document}

\maketitle
\tableofcontents


\section{Grupo 1}
\label{sec:org7b2d154}
\begin{enumerate}
\item Caracterize um sistema de aprendizagem automatica no que se respeita a:
\begin{enumerate}
\item Paradigmas de aprendizagem
\item Tecnicas de aprendizagem
\end{enumerate}

\item Tome em consideracao o dataset escolhido no trabalho pratico:
\begin{enumerate}
\item Enuncie as tecninas usadas para criar modelos de aprendizagem
\item Descreva
\item Justifique
\end{enumerate}

\item A fig seguinte representa a janela de configuracao do nodo knime ``\texttt{Partitioning}''
Descreva cada uma das opcoes
\end{enumerate}

\section{Grupo 2}
\label{sec:org268f76d}

(Estas perguntas vao ser as mais importantes, se houver muitos erros nestas o stor nao corrige o resto)

\begin{enumerate}
\item Redes Neuronais Aritificiais sao uma tecninca de aprendizagem automatica definida por uma estrutura \uline{\uline{\_\_}} de unidades computacionais denominados \uline{\uline{\_\_}}.

\item Aprendizagem com supervisao é um paradigma de aprendizagem em que os casos que  se usam para aprender contem informacao acerca dos resultados pretendidos, sendo possivel estabelecer uma relacao entre os valores pretendidos e os valores produzidos pelo sistema.
\end{enumerate}


Isto foi copiado dos slides, logo o mais certo é este tipo de perguntas virem dai e serem retiradas palavras.

\section{Grupo 3}
\label{sec:org90b12f4}
verdadeiros e falsos (corrigir a porra toda=

\begin{enumerate}
\item Uma metodologia para analise de dados descreve e creia um conjunto de passos pelos quais devera passar o desenvolvimento de um projeto de aprendizagem automatica (Machine Learning) para a resolucao de problemas
\end{enumerate}
\end{document}
