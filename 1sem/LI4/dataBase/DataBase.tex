% Created 2021-11-10 Wed 17:23
% Intended LaTeX compiler: pdflatex
\documentclass[11pt]{article}
\usepackage[utf8]{inputenc}
\usepackage[T1]{fontenc}
\usepackage{graphicx}
\usepackage{longtable}
\usepackage{wrapfig}
\usepackage{rotating}
\usepackage[normalem]{ulem}
\usepackage{amsmath}
\usepackage{amssymb}
\usepackage{capt-of}
\usepackage{hyperref}
\author{Mari§}
\date{\today}
\title{Data Base}
\hypersetup{
 pdfauthor={Mari§},
 pdftitle={Data Base},
 pdfkeywords={},
 pdfsubject={},
 pdfcreator={Emacs 27.2 (Org mode 9.6)}, 
 pdflang={English}}
\begin{document}

\maketitle
\tableofcontents

\newpage
\section{Caso de estudo}
\label{sec:orgf2ee6a8}
Com base na minha localizacao -> encontrar estacoes de comboio (transporte)

Neste caso faremos apenas em braga.

\subsection{Caracterizar estacao (Propriedades)}
\label{sec:org86664be}
\begin{itemize}
\item identificador
\item designacao
\item caracterizavao
\item endereco
\item horario
\item lista de partidas e chegadas
\item lista de opinioes
\item lista de fotografias
\item servicos disponiveis (bares, restaurantes, quiosques, \ldots{})
\item \ldots{}
\end{itemize}

\newpage

\section{Mysql}
\label{sec:orgb771d8a}
\subsection{TABELAS}
\label{sec:org59b4924}
Fazer as tabelas direito e gerar codigo (ou la o q é)
\subsubsection{ESTACOES}
\label{sec:org909e4db}
\begin{center}
\begin{tabular}{lll}
\hline
Id & CHAR(5) & PC \& NN\\
\hline
Designação & VARCHAR(5) & NN\\
\hline
Caracterização & TEXT & NN\\
\hline
Categoria & INT & NN\\
\hline
Rua & VARCHAR(100) & NN\\
\hline
Localidade & VARCHAR(75) & NN\\
\hline
Coordenadas & VARCHAR(50) & NN\\
\hline
\end{tabular}
\end{center}


\texttt{Teriamos que ligar as estacoes as categorias}


\subsubsection{CATEGORIAS}
\label{sec:org97d05f2}
\begin{center}
\begin{tabular}{lll}
\hline
Id & INT & NN \& PC\\
\hline
Designação & VARCHAR(75) & NN\\
\hline
\end{tabular}
\end{center}


\subsubsection{SERVICOS}
\label{sec:org50641cc}
\begin{center}
\begin{tabular}{lll}
\hline
Id & INT & NN \& PC\\
\hline
Designação & VARCHAR(75) & NN\\
\hline
\end{tabular}
\end{center}


\texttt{Ligamos as EstacoesServicos aos SERVICOS}

\texttt{Ligamos as EstacoesServicos as ESTACOES}


\subsubsection{EstacoesServicos}
\label{sec:org9ff61d2}
\begin{center}
\begin{tabular}{lll}
\hline
Estacao & CHAR(5) & NN \& PC\\
\hline
Servico & INT & NN \& PC\\
\hline
\end{tabular}
\end{center}


\subsubsection{NOTAS::}
\label{sec:org09de3e4}
\begin{description}
\item[{NN ->}] Nao nulo
\item[{PC ->}] Nao podem existir id iguais?!

\item[{CHAR(N) ->}] guarda sempre espaco para os N
\item[{VARCHAR(N) ->}] nao guarda sempre espaco para os N.

No caso de a string ter N/2 só guarda o espaco para esses N/2 caracteres
\end{description}
\subsection{SQL file}
\label{sec:org627005a}
FILE::

\begin{verbatim}
1  use Comboios;
2  SELECT * from ESTACOES;
\end{verbatim}

Other example
\begin{verbatim}
3  use Sakila;
4  SELECT DISTINCT store_id
5        FROM Customer
6        ORDER BY store_id;
\end{verbatim}

Na linha 4 só ira apresentar os distintos.
Se optarmos por colocar \texttt{COUNT(DISTINCT store\_id)} irá apresentar o número de resultados obtidos
\end{document}
